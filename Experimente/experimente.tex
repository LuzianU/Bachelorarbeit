\chapter{Experimente}
\label{ch:experimente}


Im Kapitel "Experimente" Ihrer Bachelorarbeit sollten Sie die durchgeführten Versuche, deren Ergebnisse und Ihre Analyse darlegen. Hier ist eine Übersicht der wichtigsten Punkte, die in diesem Kapitel enthalten sein sollten:

Versuchsaufbau:

Beschreibung der verwendeten Datensätze (Trainings-, Validierungs- und Testdaten)
Erklärung der Evaluierungsmetriken (z.B. mAP für YOLO, IoU für U-Net)
Definition der Baseline oder Vergleichsmodelle


Durchgeführte Experimente:

Detaillierte Beschreibung jedes einzelnen Experiments
Begründung für die Wahl der Experimente
Variationen in Hyperparametern, Modellarchitekturen oder Trainingsmethoden


Ergebnisse:

Präsentation der quantitativen Ergebnisse (in Tabellen oder Grafiken)
Qualitative Ergebnisse (z.B. Beispielbilder von Vorhersagen)
Vergleich der Leistung von YOLO und U-Net für Ihre spezifische Aufgabe


Analyse:

Interpretation der Ergebnisse
Diskussion von Stärken und Schwächen der Modelle
Vergleich mit dem aktuellen Stand der Technik oder anderen relevanten Arbeiten


Ablationstudie:

Untersuchung des Einflusses verschiedener Komponenten oder Hyperparameter auf die Modellleistung


Fehleranalyse:

Identifikation von häufigen Fehlertypen
Diskussion möglicher Gründe für diese Fehler


Laufzeitanalyse und Ressourcenverbrauch:

Vergleich der Inferenzzeiten von YOLO und U-Net
Analyse des Speicher- und Rechenbedarfs


Diskussion der Limitationen:

Grenzen der durchgeführten Experimente
Mögliche Verzerrungen in den Daten oder der Evaluation


Zukünftige Arbeiten:

Vorschläge für weitere Experimente oder Verbesserungen basierend auf Ihren Ergebnissen



Dieses Kapitel sollte eine objektive Darstellung Ihrer experimentellen Arbeit sein, die es dem Leser ermöglicht, die Leistung und Eignung von YOLO und U-Net für Ihre spezifische Anwendung zu verstehen und zu bewerten.