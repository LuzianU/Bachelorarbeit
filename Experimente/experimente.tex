\chapter{Experimente}
\label{ch:experimente}

Die verwendeten Metriken der Experimente fassen sich aus Präzision (precision), Erinnerung (recall), F1-Wert (F1-Score) und Richtigkeit (accuracy) zusammen. Zum Berechnen dieser wird das Aufkommen der Richtig Positiven (TP), Falsch Positiven (FP), Falsch Negativen (FN) und Richtig Negativen (TN) Vorhersagen (predictions) des Modells benutzt. Die Kategorie TP zeigt vom Modell richtig erkannte Objekte, welche tatsächlich vorhanden sind, FP bestimmt die falsche Erkennung des Modells von Objekten, die in Wirklichkeit nicht vorhanden sind, FN gibt Auskunft über Objekte, die in der Realität vorhanden sind, das Modell sie allerdings nicht erkannt hat, und TN beschreibt Objekte, die korrekt als nicht vorhanden erkannt wurden.

\[Precision = \frac{TP}{TP + FP}\]

Precision misst den Anteil der korrekten positiven Vorhersagen an allen positiven Vorhersagen des Modells. Sie zeigt, wie genau das Modell bei der Erkennung von Objekten ist und wie gut es falsche positive Ergebnisse vermeidet. Ein hoher Precision-Wert bedeutet, dass wenn das Modell ein Objekt erkennt, es mit hoher Wahrscheinlichkeit tatsächlich vorhanden ist.

\[Recall = \frac{TP}{TP + FN}\]

Recall misst den Anteil der korrekt erkannten positiven Instanzen an allen tatsächlichen positiven Instanzen. Es zeigt, wie gut das Modell alle vorhandenen Objekte einer Klasse findet. Ein hoher Recall-Wert bedeutet, dass das Modell die meisten der tatsächlich vorhandenen Objekte erkennt.

\[F1 = 2 \cdot \frac{Precision \cdot Recall}{Precision + Recall}\]

Der F1-Wert ist das harmonische Mittel aus Precision und Recall. Ein hoher F1-Wert deutet darauf hin, dass das Modell sowohl präzise als auch umfassend in seinen Vorhersagen ist. Der F1-Wert ist besonders nützlich, wenn ein ausgewogenes Verhältnis zwischen Precision und Recall wichtig ist.

\[Accuracy = \frac{TP + TN}{TP + TN + FP + FN}\]

Die Richtigkeit (Accuracy) misst den Anteil aller korrekten Vorhersagen an der Gesamtzahl der Vorhersagen. Sie gibt nur an, wie oft das Modell insgesamt richtig liegt, weshalb sie nur bei der Klassifizierung verwendet wurde.

\section{Objekterkennung}

Für das Unterkapitel der Objekterkennung wurde die Effizienz, verschiedene Diagrammarten aus Texten mithilfe der Ultralytics YOLO Objekterkennungsmodelle zu extrahieren, untersucht. Hierfür wurden die erstellten Datensätze aus \ref{ch:chartbank} zum Vortrainieren und aus \ref{ch:scanbank} zum Feintrainieren verwendet.

\subsection{Vortrainiertes Modell auf DocBank}
\subsection*{200 Trainingsbilder (mit Histogramme)}
\begin{figure}[H]
    \centering
    \captionsetup{width=1\linewidth}
    \includegraphics[width=1\textwidth]{Experimente/img/detect/1_val@0.404_200_histo/konfusionsmatrix.png}
    \caption{\hbadness=10000 x}
    \label{fig:extraction_output}
\end{figure}

\begin{table}[H]
    \centering
    \begin{tabular}{|l|c|c|c|}
        \hline
        \rowcolor[HTML]{EFEFEF}
                      & Precision        & Recall           & F1-Score         \\ \hline
        Sonstige      & 83.33\%          & 55.56\%          & 66.67\%          \\ \hline
        Linien        & 54.55\%          & 85.71\%          & 66.67\%          \\ \hline
        Balken        & 76.92\%          & 76.92\%          & 76.92\%          \\ \hline
        Histogramme   & 60.00\%          & 57.14\%          & 47.06\%          \\ \hline
        Gemischt      & 40.00\%          & 50.00\%          & 54.55\%          \\ \hline
        \textbf{Alle} & \textbf{62.96\%} & \textbf{65.07\%} & \textbf{62.37\%} \\ \hline
    \end{tabular}
    \caption{x}
\end{table}


\subsection*{200 Trainingsbilder (ohne Histogramme)}

\begin{figure}[H]
    \centering
    \captionsetup{width=1\linewidth}
    \includegraphics[width=1\textwidth]{Experimente/img/detect/2_val@0.511_200_nohisto/konfusionsmatrix.png}
    \caption{\hbadness=10000 x}
    \label{fig:extraction_output}
\end{figure}

\begin{table}[H]
    \centering
    \begin{tabular}{|l|c|c|c|}
        \hline
        \rowcolor[HTML]{EFEFEF}
                      & Precision        & Recall           & F1-Score         \\ \hline
        Sonstige      & 83.33\%          & 55.56\%          & 66.67\%          \\ \hline
        Linien        & 66.67\%          & 85.71\%          & 75.00\%          \\ \hline
        Balken        & 89.47\%          & 85.00\%          & 87.18\%          \\ \hline
        Gemischt      & 50.00\%          & 33.33\%          & 40.00\%          \\ \hline
        \textbf{Alle} & \textbf{72.37\%} & \textbf{64.90\%} & \textbf{67.21\%} \\ \hline
    \end{tabular}
    \caption{x}
\end{table}

\subsection*{321 Trainingsbilder (ohne Histogramme)}

\begin{figure}[H]
    \centering
    \captionsetup{width=1\linewidth}
    \includegraphics[width=1\textwidth]{Experimente/img/detect/3_val@0.653_nohisto/konfusionsmatrix.png}
    \caption{\hbadness=10000 x}
    \label{fig:extraction_output}
\end{figure}

\begin{table}[H]
    \centering
    \begin{tabular}{|l|c|c|c|}
        \hline
        \rowcolor[HTML]{EFEFEF}
                      & Precision        & Recall           & F1-Score         \\ \hline
        Sonstige      & 66.67\%          & 50.00\%          & 57.14\%          \\ \hline
        Linien        & 76.47\%          & 61.90\%          & 68.42\%          \\ \hline
        Balken        & 90.91\%          & 86.96\%          & 88.89\%          \\ \hline
        Gemischt      & 77.78\%          & 50.00\%          & 60.87\%          \\ \hline
        \textbf{Alle} & \textbf{77.96\%} & \textbf{62.22\%} & \textbf{68.83\%} \\ \hline
    \end{tabular}
    \caption{x}
\end{table}


\subsection{Feintraining auf historische Wirtschaftsscans}
\subsection*{2444 Trainingsbilder}
\begin{figure}[H]
    \centering
    \captionsetup{width=1\linewidth}
    \includegraphics[width=1\textwidth]{Experimente/img/detect/val@0.891 20240612-093743_double/konfusionsmatrix.png}
    \caption{\hbadness=10000 x}
    \label{fig:extraction_output}
\end{figure}

\begin{table}[H]
    \centering
    \begin{tabular}{|l|c|c|c|}
        \hline
        \rowcolor[HTML]{EFEFEF}
                      & Precision        & Recall           & F1-Score         \\ \hline
        Sonstige      & 93.75\%          & 83.33\%          & 88.24\%          \\ \hline
        Linien        & 95.83\%          & 94.52\%          & 95.17\%          \\ \hline
        Balken        & 100.0\%          & 69.57\%          & 82.05\%          \\ \hline
        Gemischt      & 100.0\%          & 76.92\%          & 86.96\%          \\ \hline
        \textbf{Alle} & \textbf{97.40\%} & \textbf{81.09\%} & \textbf{88.11\%} \\ \hline
    \end{tabular}
    \caption{x}
\end{table}


\section{Liniendiagrammsklassifizierung}
\begin{figure}[H]
    \centering
    \captionsetup{width=.75\linewidth}
    \includegraphics[width=.75\textwidth]{Experimente/img/classify/val_v1/matrix.png}
    \caption{\hbadness=10000 x}
    \label{fig:extraction_output}
\end{figure}
\begin{figure}[H]
    \centering
    \captionsetup{width=.75\linewidth}
    \includegraphics[width=.75\textwidth]{Experimente/img/classify/val_v2/matrix.png}
    \caption{\hbadness=10000 x}
    \label{fig:extraction_output}
\end{figure}

\section{Segmentation}
\subsection{Instanzsegmentation durch Ultralytics YOLO}
\begin{table}[H]
    \centering
    \begin{tabular}{|l|c|c|c|}
        \hline
        \rowcolor[HTML]{EFEFEF}
              & Precision & Recall & F1-Score \\ \hline
        Box   & 95.6\%    & 78.8\% & 86.4\%   \\ \hline
        Maske & 93.3\%    & 76.4\% & 84.0\%   \\ \hline
    \end{tabular}
    \caption{x}
\end{table}

\subsection{Semantische Segmentation durch das U-Net}
\begin{table}[H]
    \centering
    \begin{tabular}{|l|c|c|c|}
        \hline
        \rowcolor[HTML]{EFEFEF}
              & Precision & Recall  & F1-Score \\ \hline
        Maske & 91.52\%   & 89.36\% & 90.43\%  \\ \hline
    \end{tabular}
    \caption{x}
\end{table}

\section{Diagrammsauswertung}
\subsection{Achsenerkennung durch OCR}
\begin{table}[H]
    \centering
    \begin{tabular}{|c|c|c|c|}
        \hline
        \rowcolor[HTML]{EFEFEF}
        Titel & Linke Y-Achse & Rechte Y-Achse & X-Achse \\ \hline
        3     & 0             & 1              & 10      \\ \hline
    \end{tabular}
    \caption{Von den 27 Fehlern wegen fehlerhaftem Achsenerkennungsalgorithmus}
\end{table}

\begin{table}[H]
    \centering
    \begin{tabular}{|c|c|c|c|}
        \hline
        \rowcolor[HTML]{EFEFEF}
        Titel & Linke Y-Achse & Rechte Y-Achse & X-Achse \\ \hline
        6     & 6             & 4              & 10      \\ \hline
    \end{tabular}
    \caption{Von den 27 Fehlern wegen fehlerhafter optischen Schriftzeichenerkennung}
\end{table}


\subsection{Numerische Tabellenformextraktion}



Im Kapitel "Experimente" Ihrer Bachelorarbeit sollten Sie die durchgeführten Versuche, deren Ergebnisse und Ihre Analyse darlegen. Hier ist eine Übersicht der wichtigsten Punkte, die in diesem Kapitel enthalten sein sollten:

Versuchsaufbau:

Beschreibung der verwendeten Datensätze (Trainings-, Validierungs- und Testdaten)
Erklärung der Evaluierungsmetriken (z.B. mAP für YOLO, IoU für U-Net)
Definition der Baseline oder Vergleichsmodelle


Durchgeführte Experimente:

Detaillierte Beschreibung jedes einzelnen Experiments
Begründung für die Wahl der Experimente
Variationen in Hyperparametern, Modellarchitekturen oder Trainingsmethoden


Ergebnisse:

Präsentation der quantitativen Ergebnisse (in Tabellen oder Grafiken)
Qualitative Ergebnisse (z.B. Beispielbilder von Vorhersagen)
Vergleich der Leistung von YOLO und U-Net für Ihre spezifische Aufgabe


Analyse:

Interpretation der Ergebnisse
Diskussion von Stärken und Schwächen der Modelle
Vergleich mit dem aktuellen Stand der Technik oder anderen relevanten Arbeiten


Ablationstudie:

Untersuchung des Einflusses verschiedener Komponenten oder Hyperparameter auf die Modellleistung


Fehleranalyse:

Identifikation von häufigen Fehlertypen
Diskussion möglicher Gründe für diese Fehler


Laufzeitanalyse und Ressourcenverbrauch:

Vergleich der Inferenzzeiten von YOLO und U-Net
Analyse des Speicher- und Rechenbedarfs


Diskussion der Limitationen:

Grenzen der durchgeführten Experimente
Mögliche Verzerrungen in den Daten oder der Evaluation


Zukünftige Arbeiten:

Vorschläge für weitere Experimente oder Verbesserungen basierend auf Ihren Ergebnissen



Dieses Kapitel sollte eine objektive Darstellung Ihrer experimentellen Arbeit sein, die es dem Leser ermöglicht, die Leistung und Eignung von YOLO und U-Net für Ihre spezifische Anwendung zu verstehen und zu bewerten.