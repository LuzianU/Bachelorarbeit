\chapter{Methodik}
\label{ch:methodik}

\section{Extraktion von Diagrammen aus Texten}

Ziel des ersten Teils ist die Extraktion der Diagrammen aus den historischen Textscans, welche dann im folgenden Teil in eine gewünschte Form ausgewertet werden können.
\\
Die Wesentlichen Schritte des Extraktionsteils beinhalten die Objekterkennung, also die Bestimmung des Begrenzungsrechtecks (bounding box) der Diagrammen innerhalb den vorliegenden Vollseitscans und deren Unterscheidung in verschiedene Diagrammtypen, beispielsweise Linien- und Balkendiagrammen.
Die erkannten Liniendiagramme werden anschließend anhand ihrer Auswertungsschwierigkeit klassifiziert, etwa durch Kennzeichnung deren Diagrammen, welche kontextbedingt gruppiert wurden, zum Beispiel aufgrund gemeinsamer Graphsachsen.

\subsection{Erkennung von Diagrammen in Texten}
Um mit Hilfe von Deep-Learning Modelle zu trainieren, werden annotierte Grundwahrheiten (ground truth) benötigt.

\subsection*{DocBank Datensatz}

Für die Erkennung von Diagrammen in Texten wurden DocBank \cite{li2020docbank} und ein Anteil der historischen Wirtschaftsscans verwendet. DocBank besteht aus wissenschaftliche Publikation mit computergenerierten Grafiken zusammengesetzt, weshalb DocBanks Dokumentenseiten lediglich zum Vortrainieren des Detektionsmodells gedacht sind. Beabsichtigt wurde dieser Prozess des Vortrainierens um das System schneller und algemeingültiger, also mit besseren Voraussagen, trainieren zu können. Spätere Experimente untersuchen diese Annahme.
\\
An die Vorkommenshäufigkeit bei den historischen Scans angepasst, wurde die Differenzierung in fünf Objektklassen beschlossen: Linien (line), Balken (bar), Histogramm (histogram), Sonstige (other) und Gemischt (mixture). Aufgrund von Verwechslungen des Modells im Verlauf der Experimente zwischen Balkendiagrammen und Histogrammen wurden die Datensätze auf vier Klassen reduziert, indem Balkendiagramme und Histogramme vereinigt wurden.
\\
Die Schwierigkeit zwischen Balkendiagrammen und Histogrammen zu unterscheiden beruht darauf, dass Balkendiagramme kategorische Datenvergleiche anschaulich machen, bei denen die Balkenanordnung irrelevant ist, während Histogramme kontinuierliche, numerische Daten darstellen. Die Differenz liegt lediglich an der Achsenbeschreibung und nicht an visuellen Hinweisen, oftmals werden Balkendiagramme jedoch mit Lücken zwischen den Balken dargestellt, während Histogramme lückenlos abgebildet werden; dies ist allerdings nicht ausschlaggebend zur Bestimmung des Diagrammtyps.
\\
Für die manuell GT-Annotation der DocBank Dokumentenseiten, sowie folgender anderer Datensätze, wurde die Annotationssoftware CVAT \cite{CVAT_ai_Corporation_Computer_Vision_Annotation_2023} verwendet.

\begin{figure}[H]
    \centering
    \captionsetup{width=.75\linewidth}
    \includegraphics[width=.75\textwidth]{Methodik/img/docbank_example.png}
    \caption{\hbadness=10000 Beispiel kontextbedingter Gruppierung wegen gemeinsamer Y-Achsenbeschreibung eines gemischten Diagrammtyps (Histogramm und Liniendiagramm)}
    \label{fig:docbank_example}
\end{figure}

Da der Datensatz aus einer beträchtlich diversen Menge verschiedener wissenschaftlichen Publikationen besteht, beinhalten diese auch zahlreich verschiedene Diagrammlayouts. Um eine bestmögliche Konsitenz und Nützlichkeit in der Handannotation zu gewährleisten wurden einige Überlegungen gemacht: Da einige Abbildungen als Gruppe von Diagrammen fungieren (siehe Abbildung \ref{fig:docbank_example}) muss die generelle Entscheidung getroffen werden, jedes Diagramm der Gruppe einzeln zu annotieren oder lediglich die gesamte Gruppe zusammen. Beide Möglichkeiten liefern Vor- und Nachteile; beim getrennten Annotieren muss die Gruppe in einem späteren Schritt nicht mehr in die einzelnen Diagramme aufgeteilt werden, jedoch können auch kontextbedingte Informationen verloren gehen, wie in dem abgebildeten Beispiel die Y-Achsenbeschreibung des mittleren Diagramms (B), welches sich eine gemeinsame Y-Achsenbeschriftung mit dem linken Diagram (A) teilt.
Ebenfalls können Diagrammgruppen aus verschiedenen Diagrammtypen bestehen, etwa Histogramme und Liniendiagramme beieinander, weswegen dementsprechend für genau diesen Fall die gemischte Diagrammklasse eingeführt wurde. Bei weiteren Unklarheiten des Gruppenumfangs wurde sich sonst immer an die darunterliegenden Abbildungsunterschrift gehalten.
\\
Insgesamt wurden 321 Seiten annotiert, beinhaltend aus 105 Liniendiagrammen, 115 Balkendiagrammen (vereinigt mit Histogrammen), 79 sonstige und 66 gemischte Diagrammen.


\begin{wrapfigure}{r}{0.25\textwidth}
    % \vspace{-\intextsep} % Remove vertical space above the figure
    \centering
    \includegraphics[width=\linewidth]{Methodik/img/scanbank_example.png}
    \caption{\hbadness=10000 Diagrammbeispiel historischer Scans}
    \label{fig:scanbank_example}
\end{wrapfigure}
\subsection*{Datensatz historischer Wirtschaftsscans}
Die Scans der geschichtlichen Wirtschaftsmagazine wurden mit ähnlichen Überlegungen annotiert. Hier befinden sich ebenfalls Diagrammgruppen, teils auch mit mehreren verschiedenen Diagrammtypen, welche alle wieder als gesamte Gruppe annotiert wurden. Bis auf sehr wenigen Ausnahmen, befinden sich alle Abbildungen in den Scans visuell eingerahmt. Da die Ausrichtung derer jedoch nie wirklich perfekt gerade dargestellt wurde, und somit, der Ausrichtung verschuldet, kein Annotationsrechteck mit ausgeschlossenem Abbildungsrahmen gezeichnet werden kann wurde die Entscheidung getroffen, jede Annotation mit allen Ecken der Diagrammrahmen zu beinhalten.

\newpage
\section{Schwierigkeitsklassifizierung von Liniendiagrammen}
asd