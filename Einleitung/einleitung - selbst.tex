\chapter{Einleitung}
\label{ch:einleitung}

Die zunehmende Digitalisierung von Archivmaterialien ermöglicht es, historische Daten in einem nie dagewesenen Umfang zugänglich zu machen. Insbesondere die Massenextraktion und numerische Auswertung von Informationen aus historischen Dokumenten bieten ein enormes Potenzial für die Forschung. Diese Arbeit befasst sich mit der automatisierten Extraktion und Auswertung von Diagrammen aus Scans historischer Wirtschaftsmagazine. Ziel ist es, Methoden zu entwickeln und evaluieren, die eine automatische Massenerfassung numerischer Daten aus diesen Dokumenten ermöglichen.
\\
Im Fokus dabei steht die Extraktion von verschiedener Diagrammarten aus Textscans und die anschließende Tabellenformauswertung extrahierter Liniendiagrammen. Diese oft handgezeichneten und qualitativ unterschiedlichen Diagramme stellen eine besondere Herausforderung für die automatisierte Verarbeitung dar. Durch den Einsatz unterschiedlicher Deep-Learning-Methoden und speziell erstellten Datensätzen wird die Effizienz der Extraktion und die Genauigkeit der daraus resultierenden Daten analysiert.
\\
Dabei werden verschiedene Aufgabenbereiche der maschinellen Bildverarbeitung erforscht. Für die Diagrammextraktion aus den Texten werden Objekterkennungs- und Klassifikationsmodelle trainiert. Die Liniendiagrammsauswertung dagegen wird durch eine Kombination traditioneller Bildbearbeitungsmethoden, optischer Schriftzeichenerkennung und Bildsegmentierungsmodellen erforscht, welche Effizienz dieser ebenfalls anhand semantischer und Instanzsegmentation verglichen werden.
\\
Die Arbeit gliedert sich in mehrere Abschnitte, die von der grundlegenden Datenaufbereitung über die Diagrammextraktion bis hin zur Auswertung der Liniendiagramme reichen. Ziel ist es, eine robuste Methode zur Massenextraktion und -auswertung zu entwickeln, die zukünftig als Grundlage für weiterführende Studien zur Analyse historischer Wirtschaftsdaten dienen kann.